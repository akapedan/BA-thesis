%
% File: chap01.tex
% Author: Your Name
% Description: Introduction.
%
\let\textcircled=\pgftextcircled
\chapter{Introduction}
\label{chap:intro}

\begin{center}
    "Corruption is a cancer, a cancer that eats away at a citizen's faith in democracy, diminishes the instinct for innovation and creativity." \citet{quote}
\end{center} \\

\initial{J}oe Biden, former vice president of the US, blatantly analogizes the phenomenon of corruption as a malign force that attacks an otherwise healthy body - the state. Effectively, corruption can be even more of a menace. Active and deliberate use of corruption acts like a virus that spreads rapidly, contaminating societies and undermining economies. Hitherto, corruption can be found causing human suffering in every corner of the world. Particularly for people from developing countries, corruption is a part of everyday life. This is also the case for the poorest country on the European continent - Albania. As a former, highly isolated communist country, Albania has been on the transition economy train for more than 30 years now. Without neglecting that extensive and progressive work has been done to reprogram the centralized economy towards an open-market economy, various anti-corruption measures have also been taken, which, paradoxically, do not seem to reduce corruption. Instead, recent data suggests that corruption has worsen. This fact stems from data measuring corruption at the firm level. Thus, the type of corruption related to the interaction of the state with the private sector, namely bribing of public officials. There is a growing strand of literature quantitatively assessing the microeconomic effects of corruption on firm performance. More precisely, economists quarrel over two opposing theories of the effects of corruption on economic performance - on one hand the positive, greasing hypothesis and on the other hand the negative, sanding hypothesis. On these grounds, it seemed fruitful to apply this approach to the case of Albania, so that we shed light on how administrative corruption impacts Albanian firm performance. Moreover, we will test for the direct and moderating effects of bureaucratic obstacles and informal sector competition on firm performance and the relationship between corruption and firm performance, respectively. 

To the best of our knowledge, this study is among a few, which apply this micro-level approach on a single country and the first one on Albania. Furthermore, we seek to contribute to the literature by assessing four different firm performance and three different corruption variables, as well as assessing the importance to include variables measuring the direct and moderating effects of institutional quality and informal sector competition. The methodology used to measure these effects consists of Multiple Linear Regression and Logistic Regression analysis. By conducting a thorough analysis we expect to reduce a possible variable selection bias, which may improve the robustness and generalization of our findings. 

The structure of this paper is as follows: chapter 2 provides the theoretical background of our study, which elaborates a definition of corruption, reviews the literature and assess the historical development of corruption in Albania. The last section of this chapter summarizes the insights into five hypotheses. Chapter 3 describes the Business Environment and Enterprise Performance Survey (BEEPS) data-set and explains the collected dependent, independent and control variables. It also delineates the econometric models we use to test our hypotheses and visualizes the methodology. Chapter 4 verifies the model assumptions and presents the descriptive statistics as well as 
the regression results. We discuss the findings in chapter 5 and conclude with chapter 6.

%=========================================================