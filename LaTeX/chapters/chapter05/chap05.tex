%
% File: chap05.tex
% Author: Your Name
% Description: Discussion
%
\let\textcircled=\pgftextcircled
\chapter{Discussion}
\label{chap:intro}

\section{Contribution to Existing Research}
\initial{T}o the best of our knowledge this is the first micro-study on the effects of administrative corruption on Albanian firm performance. Moreover, we further investigated the possible direct and moderating effects of bureaucratic obstacles and informal competition on firm performance and its relationship with administrative corruption, respectively. Thereby, we did our best to capture the full potential of the BEEPS data-set. We thoroughly tested 12 variable combinations yielding a total of 72 regressions. Unfortunately, these models performed differently in terms of statistical significance. Hence, we restricted the above analysis to three out of nine firm growth models by picking for each corruption proxy the most insightful model. The remaining three models measuring the impact of administrative corruption on innovation activities were fully covered.

None of the regressions testing the first hypothesis showed a statistically significant positive effect of corruption on firm performance. However, three models (SB, EBI and IB) showed a significant positive effect at the 10\% level when conditioning on either policy obstacle or informal competition. Despite the coefficients' large standard errors in the remaining regressions, all of them point towards a greasing the wheels relationship between administrative corruption and firm performance. Nevertheless, these results are not significantly different from zero. However, this may be due to data-related constraints and the violation of model assumptions which will be discussed in \textbf{section 5.3}. 

Next, all of the presented firm growth models provided evidence for hypothesis 2a. Thus, policy obstacle as a proxy for bureaucratic complexity seems to be an important factor influencing sales and employment growth. Contrary to these results, the innovation activity - bribe index model showed a positive effect with a p value below 5\%. This result is both unique and strange, indicating that higher scores of policy obstacle increases the likelihood of Albanian enterprises to engage in innovation activities. 

Hypothesis 2b aimed at gauging the effects of administrative corruption on firm performance conditional on policy obstacle. Thereby, we sought to find evidence for the theoretically implied importance of assessing the effects of administrative corruption in the context of the institutional framework. This rationale is based on the theoretical studies, which developed the greasing wheels hypothesis. The regressions of equation 2b could not find a statistical significant moderating effect of policy obstacle except for model IIB, which indicates a positive interaction coefficient with a p value below 5\%. This result represents the only support for hypothesis 2b. Interestingly, the relationship between inspection bribes and innovation activities seems to be negative when firms do not experience policy obstacles. With increasing policy obstacle scores, the relationship becomes positive. Thus, corruption proxied as tax inspection bribe serves as a mechanism to bypass the bureaucratic obstacles experienced by firms when dealing with the tax administration, customs and trade regulation, business licensing and permits, and labor regulation. This is in line with the findings of \citet{goedhuys2016corruption} For robustness test reasons we investigated if we find the same result when measuring the compound policy obstacle variable solely as the degree of obstacle for tax administration. However, the results of this regression were found to be insignificant.\footnote{We do not report this regression output.} Hence, this robustness test suggests that the positive moderating effect of policy obstacle on the relationship between inspection bribe and innovation activities, though significant with a p value below 5\%, should be treated with caution. 

Furthermore, considering hypothesis 3a, we aimed to test the effect of firms experiencing informal competition on firm performance. Three out of the six presented models (i.e. models EBI, IBI and IIB) found a significant negative effect (with a confidence either below 10\% or 5\%) of informal competition on firm performance. More precisely, firms experiencing threats from the informal sector on one hand have a lower employment growth rate than firms which do not experience these threats (model EBI) and on the other hand are less likely to innovate compared to firms which do not compete against informal firms (models IBI and IIB). These results provide partial validation of hypothesis 3a. 

For our last hypothesis we argued that informal firms may be partially a product of public officials seeking bribes. By hiding their activity, informal firms incur a cost advantage compared to formal firms. Hence, we hypothesized that informal competition negatively moderates the relationship between administrative corruption and firm performance. EBI and IB were the only two models validating this hypothesis at a statistical significance of 10\%. This outcome is not consistent with a previous study of \citet{xie2019corruption}. Instead of raising the costs of firms, they argued that corruption acts as a competitive strategy, which decreases the threats from informal sector firms. In line with this argument, their results revealed a positive moderating effect of informal competition on the relationship between corruption and new product innovation.

In summary, our regression analysis yielded the following insights: policy obstacle has a negative impact on sales growth (model SB) and employment growth (model EBI and EIB), and a positive effect on the likelihood to be an innovator (model IBI). Moreover, policy obstacle positively moderates the relationship between inspection bribe and firm innovation (model IIB). Informal competition negatively affects employment growth (model EBI) and the likelihood to be an innovator (models IBI and IIB). Lastly, the hypothesized negative moderating effect of informal competition was found to affect the employment growth - bribe index, as well as the innovation index - bribes relationship. 

\section{Limitations}
Like in any study, the present thesis has its limitations.  First, the overall insignificant effect of administrative corruption on firm performance may stem from missing survey answers, which substantially diminished our sample size. This is ultimately due to the nature of our variables. On one hand, according to the World Bank in 2009, Albanian enterprises usually operate with two balance sheets in order to evade taxes.\footnote{Implementation note of BEEPS survey from 2009.} Thus, besides the fact that approximately 16\% of the surveyed firms do no report sales figures at all, by assuming that firms still operate with two balance sheets in 2019, it is plausible that Albanian firms under-report their sales figures. On the other hand, corruption may be under-reported due to its illicit and sensitive nature. Thus, firms may not be comfortable answering questions on corruption. 

A further limitation, which needs to be addressed concerns the approximative measurement of our variables of interest. Even though we did our best to capture the full potential of the BEEPS data-set by using different measures, these approximations can not cover all components of administrative corruption. These constraints indicate that our models may be subject to measurement error.

Another limitation relates to a selection bias. The share of rejection per contacted firm was 20.3\%. Thus, every fifth contacted firm refused to participate. However, all enterprise surveys suffer from this shortcoming. Compared to the other countries in the region Albania is among the ones with the lowest rejection rate. 

Furthermore, the partially weak results of our models may also be a reflection of the violation of the model assumptions. Particularly, the violation of the linearity assumption puts forth a major drawback on our OLS estimations. Finally, we did not control for the possible endogeneity issue between corruption and firm performance.

Due to these constraints on data size and quality, as well as the methodological shortcomings, the validity of the analyses is limited. Irregardless of these limitations, it is also possible that administrative corruption does not have an impact on Albanian firm performance that is statistically different from zero - as has been suggested by our results. 

\section{Future Research}
The present study sheds light on the difficulty of measuring the effect of administrative corruption on firm performance in a single and small country. Since there exist only a few studies analyzing firm-level corruption in a single country there is plenty of room for future studies. In order to yield more robust results researches should focus on different methodologies to circumvent the data-related issues. Moreover, we posit that upcoming research should use more rigorous and sophisticated analysis. Especially the possible endogeneity concern between corruption and firm performance is a fruitful avenue for future investigations since previous studies often lack a statistically sound approach. 

%\section{Personal Learnings}

