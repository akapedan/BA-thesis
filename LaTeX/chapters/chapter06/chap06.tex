%
% File: chap06.tex
% Author: Your Name
% Description: Conclusion
%
\let\textcircled=\pgftextcircled
\chapter{Conclusion}
\label{chap:results}

\initial{T}he present study intended to contribute to the firm-level corruption literature by analyzing the effect of administrative corruption on firm performance in a single country - Albania. Thereby, we aimed to gain insights into the micro-processes of the state-firm relationship whether corruption greases or sands the wheels of commerce. We first attempted to provide a description of the phenomenon of corruption and the various forms it takes. Thereafter, we conducted a thorough literature analysis in order to educate ourselves and the readers on the past research efforts, which assessed the economic implications of corruption. The penultimate section was devoted to give a concise summary of the historical development of corruption in Albania. The last section of the theoretical background summarized the accumulated insights into five research hypotheses. The first hypothesis stated that administrative corruption has a positive effect on Albanian firm performance. Hypothesis 2a stated that bureaucratic complexity has a negative direct effect on the performance of Albanian firms. In order to test the interaction of bureaucratic complexity and administrative corruption we formulated in hypothesis 2b that the level of bureaucratic complexity positively moderates the relationship between administrative corruption and Albanian firm performance. Next, in hypothesis 3a we declared that informal competition has a negative direct effect on firm performance. Finally, the last hypothesis (3b) stated that informal competition negatively moderates the relationship between administrative corruption and firm performance. Chapter three described the BEEPS data-set, the variables, which we extracted from it and the methodological approach to test our hypotheses. In order to thoroughly test for our predictions, we conducted in total 72 regression by using four different dependent variables (firm performance) and three different key independent variables (corruption), which yielded 12 combinations \`{a} six regression equations. More precisely, we constructed three firm growth variables, which measure (1) annual sales growth, (2) annual employment growth and (3) annual labor productivity, and we constructed one firm innovation variable, which measures the innovation activity of firms. We used the Multiple Linear Regression Model for the regressions, which incorporate a firm growth variable and due to the binary nature of the firm innovation variable we used the Logistic Regression Model to gauge the effects. In chapter four we verified the model assumptions as well as presented the descriptive statistics and regression results. Unfortunately, we had to restrict the presentation of the nine firm growth models to three of the best performing with respect to the corruption proxies. Hence, we presented the sales growth - bribe (SB) model, the employment growth - bribe index (EBI) model, the employment growth - inspection bribe (EIB) model and the three innovation index models (IB, IBI and IIB). Our empirical findings lead us to reject our first hypothesis for all the presented models. Hypothesis 2a is line with the presented firm growth models but not with the innovation models. Next, only model IIB confirmed our hypothesis 2b. Thus, bureaucratic obstacles positively moderates the initial negative relationship between inspection bribe. This finding provides evidence for the greasing hypothesis in dependency of the institutional quality. Furthermore, the models EBI, IBI and IIB found evidence in favor of hypothesis 3a. This means that informal competition has a negative impact on employment growth and innovation activities of Albanian enterprises. Lastly, hypothesis 3b is validated by models EBI and IB, which indicates that informal competition negatively moderates the relationship between bribe index and employment growth, as well as between bribes and innovation activities. We mainly attributed the weak results 
to data-related issues, which lead to the violation of the model assumptions. More precisely, the low amount of data available for this study as well as the potential under-reporting and measurement error of our key variables limited and biased our findings. Therefore, our study further showed the difficulty of measuring the effects of firm-level corruption in a single and small country. Our study ended with a discussion about the contributions and limitations of this study, as well as with future research recommendations. Since the results must be interpreted carefully we omitted the policy recommendations. 

%=======
